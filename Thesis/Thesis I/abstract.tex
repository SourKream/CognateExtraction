\begin{center}
\LARGE{Abstract}
\end{center}

\vspace{0.5in}

The high level of linguistic diversity in South Asia poses the challenge of building lexical resources across the languages from these regions. This project is aimed at automatically discovering cognates between closely related language pairs like Hindi-Marathi or Hindi-Punjabi, in a scalable manner, which would assist the task of lexical resource creation. We would like to analyze a large part of the vocabulary for both languages, as opposed to small word lists used in most works so far. We also aim to do a linguistic analysis over the identified cognates to conclude whether lexical resources can be successfully shared between Hindi and related languages.
