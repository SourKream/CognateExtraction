\chapter{Introduction}

\section{Motivation}

Cognates are words across different languages that are known to have originated from the same word in a common ancestral language. For example, the English word `\textit{Night}' and the German `\textit{Nacht}', both meaning \textit{night} and English `\textit{Hound}' and German `\textit{Hund}', meaning \textit{dog} are cognates whose origin can be traced back to Proto-Germanic \cite{rama2015automatic}. Cognates are not always revealingly similar and can change substantially over time such that they do not share form similarity. The English word `\textit{wheel}' and the Sanskrit word `\textit{chakra}' are in fact cognates which are traced back to `$*k^wek^welo$' from Proto-Indo-European.

Automatic cognate identification, in Natural Language Processing, refers to the application of string similarity algorithms with machine learning algorithms for determining if a given word pair is cognate or not. Identification of cognates is essential in historical linguistics. Cognate information has also been successfully applied to NLP tasks, like sentence alignment in bitexts \cite{simard1993using}, improving statistical machine translation models \cite{kondrak2003cognates}, inducing translation lexicons \cite{mann2001multipath}\cite{tufis2002cheap} and identification of confusable drug names \cite{kondrak2004identification}. It can also be used to bootstrap lexical resource creation for a language with low resources by finding parallels in related rich resource languages.

\section{Outline}

In this work, we have performed rigorous analysis over the existing state of the art models used for cognate identification to reveal their short comings and pitfalls and suggest possible improvements over these models to be implemented and tested in the second part of the project. 

In the following chapters, we first describe the previous works in the field of automatic cognate discovery, the datasets used in the work, the experiments conducted along with the results and error analysis of the different models and the future work that we have planned motivated by our results.

