\begin{center}
\LARGE{Abstract}
\end{center}

\vspace{0.5in}

The high level of linguistic diversity in South Asia poses the challenge of building lexical resources across the languages from these regions. This work is aimed at automatically discovering cognates between closely related language pairs like Hindi-Marathi. In this work, we present a deep learning model for this task of pairwise cognate prediction. We use a character level model with recurrent architecture and attention that has previously been employed on several NLP tasks. We compare the performance of our model with previous approaches on various language families. We also employ our model specifically to the domain of discovering cognates between Hindi and Marathi, which would assist the task of lexical resource creation. We analyze a large part of the vocabulary for both languages, as opposed to small word lists used in most works so far.
